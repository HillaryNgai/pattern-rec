\documentclass{article}

\usepackage{graphicx} % Required for the inclusion of images
\usepackage{amsmath} % Required for some math elements
\topmargin0.0cm
\headheight0.0cm
\headsep0.0cm
\oddsidemargin0.0cm
\textheight23.0cm
\textwidth16.5cm
\footskip1.0cm

\title{Lab 1 Report} % Title
\author{SYDE 372} % Author name
\date{February 12th, 2017} % Date for the report

\begin{document}

\maketitle

\begin{center}
\begin{tabular}{l r}
Group Members: & Krishn Ramesh - 20521942 \\ % Partner names
& Brady Kieffer - 20517665 \\
& Ramandeep Farmaha - 20516974 \\
& Shubam Metha - 20483061\\
Instructor: & Professor Wong % Instructor/supervisor
\end{tabular}
\end{center}


%----------------------------------------------------------------------------------------
%   SECTION 1
%----------------------------------------------------------------------------------------

\section{Introduction}

Clustering and classifying data points is a fundamental skill in pattern recognition. Lab 1 provides a hands-on introduction to generating different kinds of decision boundaries, as well as evaluating and comparing their performance. Two cases with true statistics (mean, covariance matrices) for the classes in each case were provided. The initial setup work in generating the sampled data involved bivariate normal distributions subject to orthonormal transformations. Different distance measures were used to construct decision boundaries including MED, GED, MAP, NN and kNN. Lastly, the performance of each distance measure was analyzed and compared. Overall, Lab 1 provided valuable insight into the use of orthonormal transformations, decision boundaries and classification errors to solve practical problems.


\end{document}